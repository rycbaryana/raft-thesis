\chapter*{\large Введение}  
\addcontentsline{toc}{chapter}{Введение}
В современную эпоху цифровой трансформации наблюдается фундаментальный сдвиг парадигмы построения программного обеспечения от монолитных архитектур к распределенным системам. С ростом популярности облачных вычислений, микросервисной архитектуры и технологий обработки больших данных (Big Data), критически важной становится проблема обеспечения надежности и доступности сервисов. Однако физическая природа распределенных сетей диктует суровые условия: узлы могут выходить из строя, пакеты данных могут теряться или задерживаться, а сетевые сегменты могут изолироваться друг от друга. В таких условиях обеспечение согласованности данных между разрозненными узлами становится нетривиальной алгоритмической задачей.

Центральным механизмом, позволяющим группе ненадежных вычислений действовать как единое целое, является распределенный консенсус. Алгоритмы консенсуса гарантируют, что все исправные серверы в кластере придут к согласию относительно последовательности команд или состояния данных, даже в присутствии сбоев. Долгое время «золотым стандартом» в этой области считался алгоритм Paxos, предложенный Лесли Лэмпортом. Однако высокая концептуальная сложность Paxos и трудности его корректной программной реализации привели к необходимости поиска альтернативных решений.

Ответом на этот вызов стал алгоритм Raft, представленный в 2014 году Диего Онгаро и Джоном Оустерхаутом. Ключевой особенностью Raft является его ориентированность на понятность без ущерба для корректности и производительности. Сегодня Raft лежит в основе таких критически важных систем индустриального уровня, как etcd (хранилище конфигураций Kubernetes), Consul и CockroachDB.

Актуальность темы курсовой работы обусловлена необходимостью глубокого понимания механизмов отказоустойчивости, лежащих в основе современной инфраструктуры. Разработка собственной реализации алгоритма распределенного консенсуса является наилучшим способом изучения проблем синхронизации состояний, обработки сетевых коллизий и проектирования RPC-протоколов. Более того, язык программирования Go, выбранный для реализации, де-факто является стандартом для разработки облачных и распределенных систем благодаря своим примитивам конкурентности.

Целью данной работы является проектирование и программная реализация системы распределенного консенсуса на основе алгоритма Raft, обеспечивающей коррекцию реплицированного лога команд в условиях частичных отказов инфраструктуры.