\chapter*{ЗАКЛЮЧЕНИЕ}
\addcontentsline{toc}{chapter}{Заключение}

В ходе выполнения курсовой работы была исследована фундаментальная проблема достижения распределенного консенсуса в условиях ненадежной аппаратной среды и разработан действующий программный прототип системы на основе алгоритма Raft.

В процессе работы были успешно решены следующие задачи:

\begin{enumerate}
    \item \textbf{Выполнен теоретический анализ предметной области.}
    Рассмотрены ключевые модели распределенных вычислений и ограничения разрешимости (теоремы FLP и CAP). Формализована задача построения Реплицированного конечного автомата (RSM). Обоснован выбор алгоритма Raft для построения надежных CP-систем, обладающего преимуществом в понятности перед алгоритмом Paxos.

    \item \textbf{Детально изучена алгоритмическая модель Raft.}
    Проанализированы механизмы обеспечения безопасности (Safety) и живучести (Liveness). Разобраны процедуры выбора лидера, репликации журнала и восстановления согласованности. Показано, как использование модели сильного лидерства и рандомизированных таймеров позволяет эффективно решать проблему разделения голосов.

    \item \textbf{Спроектирован и реализован программный комплекс.}
    Разработана архитектура распределенного приложения на языке Go. Реализация включает в себя транспортный слой на базе RPC, ядро консенсуса с управлением состояниями и механизм обработки клиентских запросов.

    \item \textbf{Проведена верификация системы.}
    Выполнено интеграционное тестирование разработанного прототипа. Экспериментально подтверждена способность системы корректно избирать лидера, реплицировать данные и автоматически восстанавливать целостность кластера после сбоев узлов.
\end{enumerate}

\textbf{Практическая значимость} работы заключается в создании документированной, модульной реализации алгоритма консенсуса, которая может служить фундаментом для построения отказоустойчивых распределенных хранилищ или сервисов координации микросервисов.

В качестве направлений для дальнейшего развития проекта можно выделить реализацию механизма сжатия журнала посредством создания моментальных снимков и добавление поддержки динамического изменения конфигурации кластера без остановки обслуживания.
